\documentclass[a4paper, 11pt]{exam}
\usepackage{amsmath}
\usepackage{amssymb}
\usepackage{amsthm}
\usepackage[italian]{babel}
\usepackage{ccicons}
\usepackage{hyperref}
\usepackage{cleveref}
\usepackage[utf8]{inputenc}
\usepackage[autostyle=false, style=english]{csquotes}
\usepackage[margin=2cm]{geometry}
\usepackage{graphicx}
\usepackage{mathrsfs}
\usepackage{multicol}
\usepackage{relsize}
\usepackage{parskip}
\usepackage{titling}
\pagestyle{plain}
\graphicspath{{./images/}}
\MakeOuterQuote{"}
\setlength{\columnseprule}{.4pt}
\renewcommand{\solutiontitle}{\noindent\textbf{R:}\enspace}
\renewcommand{\familydefault}{\sfdefault}
\def\dbar{{\mathchar'26\mkern-12mu d}}
\title{Verifica di Informatica - Funzioni}
\author{ITIS Aldini-Valeriani - Classe 3EIN}
\date{15 aprile 2021}
\begin{document}
\maketitle
\begin{center}\fbox{\fbox{\parbox{5.5in}{

		\centering
		
		Si consegni su Classroom \textbf{un solo file} con \textbf{nome nella forma} \texttt{Nome\_Cognome\_2021-04-15.cpp}.

}}}

\end{center}
%\vspace{1cm}
%Nome: \enspace \hrulefill
\vspace{1em}

\section*{Problema}

L'azienda di consulenza informatica GrandiSpioni è interessata ad ampliare la propria offerta di prodotti di \emph{consumer analytics} integrando nel proprio prodotto di punta, SpiaLaSpesa, una API (\emph{Application Programming Interface}) chiamabile mediante RPC (\emph{Remote Procedure Call}). 

SpiaLaSpesa permette, tra le altre cose, di condurre indagini di mercato suddividendo in categorie il pubblico consumatore e tenendo traccia del consumo di più prodotti da parte delle varie categorie. 
Un esempio di possibile struttura del \emph{database} della GrandiSpioni è mostrato nella tabella seguente, nella quale le celle rappresentano le unità di prodotto vendute a ogni categoria e dalla quale appare chiara la correlazione tra altezza delle persone e interessi in àmbito artistico, nonché tra età e interesse nella schematizzazione. 

\begin{table}[h!]

	\centering

\begin{tabular}{|l|l|l|l|l|}
\hline
                            & \textbf{Alte 18-25} & \textbf{Basse 18-25} & \textbf{Alte 26-40} & \textbf{Basse 26-40} \\ \hline
\textbf{Libri}              & 8                           & 33                           & 52                          & 98                           \\ \hline
\textbf{Strumenti musicali} & 87                          & 17                           & 9                           & 90                           \\ \hline
\textbf{Lavagne}            & 29                          & 34                           & 91                          & 61                           \\ \hline
\end{tabular}
\end{table}

In qualità di \textit{junior software engineer} presso GrandiSpioni ti è richiesto di realizzare una serie di funzioni che assolvano ai requisiti e del cui \emph{deployment} sotto forma di \emph{endpoint} si occuperà il \emph{team DevOps}. 
Per verificare la corretta esecuzione delle funzioni implementate  ti è richiesto di realizzare una \texttt{main} di esempio che le chiami e le testi su dei dati inseriti da riga di comando.
Il programma di verifica dovrà essere realizzato nel linguaggio C++, implementare una \texttt{main} di prova che presenti un'interfaccia da linea di comando e assolvere ai requisiti indicati di séguito.

\begin{questions}
	\question esporre una funzione \texttt{getIncomeFromProductCategory()} che, presi in input i dati di vendita, le categorie di prodotto e di clientela, i prezzi di ogni prodotto e una categoria di prodotto, sommi le unità vendute del prodotto indicato, le moltiplichi per il prezzo relativo e restituisca gli incassi provenienti da quel prodotto (2 punti)
	\question esporre una funzione \texttt{getIncomeFromConsumerCategory()} che, presi in input i dati di vendita, le categorie di prodotto e di clientela, i prezzi di ogni prodotto e una categoria di clientela, moltiplichi le unità di ogni prodotto vendute alla categoria di clientela indicata per il prezzo relativo, sommi i risultati e restituisca gli incassi riferibii alla categoria di clientela indicata (2 punti)
	\question esporre una funzione \texttt{getProductIncome()} che si serva della  \texttt{getIncomeFromProductCategory()} per riempire un array in modo che, dopo l'esecuzione della funzione, contenga gli incassi riferibili a tutte le diverse categorie di prodotto (2 punti)
	\question esporre una funzione \texttt{getConsumerIncome()} che si serva della  \texttt{getIncomeFromConsumerCategory()} per riempire un array in modo che, dopo l'esecuzione della funzione, contenga gli incassi riferibili a tutte le diverse categorie di pubblico consumatore (2 punti)
	\question implementare una funzione \texttt{sortAscending()} che ordini un array in ordine crescente (2 punti)
	\question esporre due funzioni \texttt{getSortedProductIncome()} e \texttt{getSortedConsumerIncome()}, che riempiano un array passato come argomento in modo che contenga, \emph{ordinati}, i dati restituiti da \texttt{getProductIncome()}  e  \texttt{getConsumerIncome()}. Non è necessario mantenere l'associazione tra valori e categorie, ma solo restituire l'array ordinato. (2 punti)
\end{questions}

La manager del dipartimento di Servizi Informatici della GrandiSpioni indica che le categorie di prodotto e di clientela, così come le dimensioni del database, possono essere fissate nel codice o esternalizzate e richieste in input. 
In quest'ultimo caso, come compensazione per il lavoro straordinario, è previsto un bonus di 1 punto.
La manager ricorda anche che, nel caso vi siano problemi l'implementazione della \texttt{sortAscending()}, è possibile ordinare un array \texttt{arr} di dimensione \texttt{n} aggiungendo all'intestazione del codice la direttiva \texttt{\#include <algorithm>} e servendosi della funzione \texttt{sort(arr, arr + n)}.

Un esempio di esecuzione del software è mostrato nella sezione seguente. 
Le linee che iniziano con il carattere \texttt{>} rappresentano l'input degli utenti.
Chiaramente, l'interfaccia mostrata rappresenta un mero \textit{mock-up} esemplificativo e non è assolutamente vincolante. 

Infine, la manager si raccomanda di scrivere codice pulito, funzionale e ben indentato, nel quale sia chiaro lo scopo e il funzionamento di ogni riga.
\section*{Esempio di esecuzione}

\begin{verbatim}
Benvenuto/a in GrandiSpioni!

Quante categorie di prodotto sono previste? 
> 3

Inserire la categoria di prodotto 0.
> Libri
Inserire il prezzo del prodotto 0.
> 2
Inserire la categoria di prodotto 1.
> Strumenti
Inserire il prezzo del prodotto 1.
> 20
Inserire la categoria di prodotto 2.
> Lavagne
Inserire il prezzo del prodotto 2.
> 10

Quante categorie di pubblico consumatore sono previste? 
> 4

Inserire la categoria di pubblico 0.
> Basse1825
Inserire la categoria di pubblico 1.
> Alte1825
Inserire la categoria di pubblico 2.
> Basse2640
Inserire la categoria di pubblico 3.
> Alte2640

Quante unità di prodotto Libri sono state vendute 
alla categoria Basse1825?
> 8
Quante unità di prodotto Libri sono state vendute 
alla categoria Alte1825?
> 33
Quante unità di prodotto Libri sono state vendute 
alla categoria Basse2640?
> 52
Quante unità di prodotto Libri sono state vendute 
alla categoria Alte2640?
> 98

Quante unità di prodotto Strumenti sono state vendute 
alla categoria Basse1825?
> 87
Quante unità di prodotto Strumenti sono state vendute 
alla categoria Alte1825?
> 17
Quante unità di prodotto Strumenti sono state vendute 
alla categoria Basse2640?
> 9
Quante unità di prodotto Strumenti sono state vendute 
alla categoria Alte2640?
> 90

Quante unità di prodotto Lavagne sono state vendute 
alla categoria Basse1825?
> 29
Quante unità di prodotto Lavagne sono state vendute 
alla categoria Alte1825?
> 34
Quante unità di prodotto Lavagne sono state vendute 
alla categoria Basse2640?
> 91
Quante unità di prodotto Lavagne sono state vendute 
alla categoria Alte2640?
> 61

Inserire la categoria di prodotto per la chiamata di test 
alla getIncomeFromProductCategory().
0 = Libri; 1 = Strumenti; 2 = Lavagne
> 0
Alla categoria Libri è riferibile un incasso di 382.

Inserire la categoria di pubblico consumatore per la chiamata di test 
alla getIncomeFromConsumerCategory().
0 = Basse1825; 1 = Alte1825; 2 = Basse2640; 3 = Alte2640
> 2
Alla categoria Basse2640 è riferibile un incasso di 1194.

Risultato della chiamata di test alla getProductIncome():
{382, 4060, 2150}

Risultato della chiamata di test alla getConsumerIncome():
{2046, 746, 1194, 2606}

Risultato della chiamata di test alla getSortedProductIncome():
{382, 2150, 4060}

Risultato della chiamata di test alla getSortedConsumerIncome():
{746, 1194, 2046, 2606}

A presto!

\end{verbatim}


\end{document}
