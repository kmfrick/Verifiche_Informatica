\documentclass[a4paper, 11pt]{exam}
\usepackage{amsmath}
\usepackage{amssymb}
\usepackage{amsthm}
\usepackage[italian]{babel}
\usepackage{ccicons}
\usepackage{hyperref}
\usepackage{cleveref}
\usepackage[utf8]{inputenc}
\usepackage[autostyle=false, style=english]{csquotes}
\usepackage[margin=2cm]{geometry}
\usepackage{graphicx}
\usepackage{mathrsfs}
\usepackage{multicol}
\usepackage{relsize}
\usepackage{parskip}
\usepackage{titling}
\pagestyle{plain}
\graphicspath{{./images/}}
\MakeOuterQuote{"}
\setlength{\columnseprule}{.4pt}
\renewcommand{\solutiontitle}{\noindent\textbf{R:}\enspace}
\renewcommand{\familydefault}{\sfdefault}
\def\dbar{{\mathchar'26\mkern-12mu d}}
\title{Verifica di Informatica - Funzioni}
\author{ITIS Aldini-Valeriani - Classe 3EIN}
\date{29 aprile 2021}
\begin{document}
\maketitle
\begin{center}\fbox{\fbox{\parbox{5.5in}{

		\centering
		
		Si consegni su Classroom \textbf{un solo file} con \textbf{nome nella forma} \texttt{Nome\_Cognome\_2021-04-29.cpp}.

}}}

\end{center}
%\vspace{1cm}
%Nome: \enspace \hrulefill
\vspace{1em}

\section*{Problema}

Si è conclusa la selezione scolastica delle Olimpiadi di Informatica e dopo le correzioni si procede all'ammissione alla prova successiva. Ogni studente viene valutato con un numero compreso tra 1 e 30.
Ogni scuola ammette gli studenti e le studentesse secondo i seguenti criteri
\begin{itemize}
	\item ogni scuola ha diritto a fare partecipare almeno 2 studenti o studentesse.
	\item lo studente o la studentessa riceve una valutazione maggiore o uguale a 20. In questo caso, tutti gli studenti e tutte le studentesse meritevoli possono partecipare.
  \item nella scuola non ci sono 2 studenti o studentesse con valutazione maggiore o uguale a 20. In questo caso partecipano al massimo 2 studenti o studentesse solo se la valutazione è maggiore della media nazionale.
\end{itemize}
Il programma riceve in input:

\begin{itemize}
	\item	il numero $N$ di studenti e studentesse che hanno partecipato, in tutta Italia, alle Olimpiadi di Informatica
	\item un elenco di $N$ valutazioni (nazionali). Le valutazioni possono essere ripetute
	\item 	il numero $M$ di studenti e studentesse della scuola Aldini-Valeriani che ha partecipato alle Olimpiadi
	\item un elenco di $M$ studenti e studentesse  della scuola Aldini-Valeriani che ha partecipato alla gara
	\item un elenco delle $M$ valutazioni ottenute da ogni studente o studentessa della scuola
\end{itemize}

il programma stampa:
\begin{itemize}
	\item $N_{ammessi}$ 	il numero di studenti o studentesse \emph{della scuola Aldini-Valeriani} ammessi alla fase successiva
	\item La classifica degli ammessi (nome e punteggio)
	\item $V_m$, la media \emph{nazionale}
\end{itemize}

Si implementino:

\begin{itemize}
	\item Una funzione per calcolare la media nazionale (2 punti)
	\item Una funzione per calcolare quanti studenti o studentesse delle Aldini hanno ottenuto un punteggio maggiore o uguale a 20 (2 punti)
	\item Una funzione per calcolare quanti studenti o studentesse delle Aldini hanno ottenuto un punteggio superiore alla media nazionale (2 punti)
	\item Una funzione per ordinare i punteggi degli studenti e delle studentesse delle Aldini in ordine decrescente di punteggio. Mantenere la corrispondenza tra punteggio e nome di ogni studente o studentessa (2 punti)
	\item Una funzione per calcolare quante studenti e studentesse provenienti dalle Aldini parteciperanno alla gara regionale (2 punti)
	\item Una funzione per stampare nomi e punteggi degli studenti e delle studentesse ammesse alla gara regionale in ordine di classifica (2 punti)
\end{itemize}

Di seguito vengono proposti due esempi di realizzazione del programma, entrambi validi. 

\section*{Esempio 1}

\begin{verbatim}
Quanti studenti e studentesse hanno partecipato alle Olimpiadi?
> 10
Quanti studenti e studentesse della scuola Aldini-Valeriani 
hanno partecipato alle Olimpiadi?
> 5
Inserire le 10 valutazioni tra i concorrenti e le concorrenti nazionali.
> 20 25 25 5 6 10 23 24 21 12
Inserire il nome dello studente o della studentessa Aldini 0
> Rossi 
Inserire la valutazione di Rossi
> 12
Inserire il nome dello studente o della studentessa Aldini 1
> Bianchi
Inserire la valutazione di Bianchi
> 20
Inserire il nome dello studente o della studentessa Aldini 2
> Verdi
Inserire la valutazione di Verdi
> 22
Inserire il nome dello studente o della studentessa Aldini 3
> Gialli
Inserire la valutazione di Gambini
> 21
Inserire il nome dello studente o della studentessa Aldini 4
> Bordeaux 
Inserire la valutazione di Bordeaux
> 19

Sono state ammesse 3 persone dalla scuola Aldini-Valeriani, secondo la seguente classifica:
Verdi 22
Gialli 21
Bianchi 20

La media nazionale è 17.1
3 studenti hanno una valutazione >= 20 e quindi partecipano
\end{verbatim}


\section*{Esempio 2}
\begin{verbatim}
Inserire i dati di input.
> 12  4
> 20 15 25 5 6 10 13 24 21 12  9 8
> Rossi
> Bianchi
> Verdi
> Bordeaux
> 6 16 12 11

Dati di output:
1
Bianchi 16

La media nazionale è 14.2
Nessuno studente e nessuna studentessa ha valutazione >= 20
1 studente ha valutazione maggiore della media nazionale e quindi partecipa
\end{verbatim}


 


\end{document}
