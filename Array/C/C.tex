\documentclass[a4paper, 11pt]{exam}
\usepackage{amsmath}
\usepackage{amssymb}
\usepackage{amsthm}
\usepackage[italian]{babel}
\usepackage{ccicons}
\usepackage{hyperref}
\usepackage{cleveref}
\usepackage[utf8]{inputenc}
\usepackage[autostyle=false, style=english]{csquotes}
\usepackage[margin=2cm]{geometry}
\usepackage{graphicx}
\usepackage{mathrsfs}
\usepackage{multicol}
\usepackage{relsize}
\usepackage{parskip}
\usepackage{titling}
\pagestyle{plain}
\graphicspath{{./images/}}
\MakeOuterQuote{"}
\setlength{\columnseprule}{.4pt}
\renewcommand{\solutiontitle}{\noindent\textbf{R:}\enspace}
\def\dbar{{\mathchar'26\mkern-12mu d}}
\title{Verifica di Informatica - Array - Compito C}
\author{ITIS Aldini-Valeriani - Classe 3EIN}
\date{26 gennaio 2021}
\begin{document}
\maketitle
\begin{center}\fbox{\fbox{\parbox{5.5in}{

		\centering
	
	Si consegni su Classroom \textbf{un solo file} con \textbf{nome nella forma} \texttt{Nome\_Cognome\_2021-01-26\_CompitoC.cpp}.


}}}

\end{center}
%\vspace{1cm}
%Nome: \enspace \hrulefill
\vspace{1em}

\section*{Problema}

La concessionaria FastCars è stata sommersa di ordini a causa dell'elevato numero di rider che hanno preso servizio durante la pandemia di COVID-19.

Per far fronte all'elevato numero di ordini la FastCars ha deciso di concentrare la produzione su veicoli potenti ed eseguire un'analisi \textit{data-driven} della produzione.

La FastCars offre quattro tipi di veicoli: motorini sbloccati\footnote{Non fatelo a casa!}, motocicli, automobili e camion.
Ognuno di questi tipi è identificato da un codice: 1 per i motorini, 2 per i motocicli, 3 per le automobili e 4 per i camion.

In qualità di \textit{junior software engineer} presso FastCars ti è richiesto di realizzare un software che si interfacci con il sito italiano della compagnia.
Il programma dovrà essere realizzato nel linguaggio C++, presentare un'interfaccia da linea di comando e assolvere ai \textit{requisiti funzionali} indicati di séguito.

\begin{questions}
	\question Leggere da tastiera le cilindrate e le tipologie di veicoli. (1 punto)
	\question Eliminare tutti i camion. Va bene, ci sono tanti ordini, ma non \textit{così} tanti! (2.5 punti)
	\question Calcolare la cilindrata massima $P_{max}$ e la cilindrata media $\bar{P}$, da fornire ai \textit{data analyst} della FastCars affinché possano effettuare le loro analisi di produzione.  (2.5 punti)
	\question Calcolare la cilindrata minima $P_{min}$ ed eliminare tutti i veicoli con cilindrata pari o inferiore a $2 \cdot P_{min}$, in accordo con la politica della FastCars. (4 punti)
\end{questions}

La manager del dipartimento di Servizi Informatici della FastCars specifica che ogni veicolo è identificato da un codice numerico non negativo, ma che non è importante che il software mantenga gli stessi codici. 
Chiaramente, il software non deve cancellare veicoli che devono essere venduti, o inventarne di nuovi!

Un esempio di esecuzione del software è mostrato nella sezione seguente. 
Le linee che iniziano con il carattere \texttt{>} rappresentano l'input degli utenti.
Chiaramente, l'interfaccia mostrata rappresenta un mero \textit{mock-up} esemplificativo e non è assolutamente vincolante. 

Infine, la manager si raccomanda di scrivere codice pulito, funzionale e ben indentato, nel quale sia chiaro lo scopo e il funzionamento di ogni riga.

\pagebreak
\maketitle
\section*{Esempio di esecuzione}

\begin{verbatim}

Benvenuto/a in FastCars! 

Quanti sono i veicoli da elaborare?
> 4

Inserire la tipologia del veicolo 0.
> 2
Inserire la cilindrata del veicolo 0.
> 130.50

Inserire la tipologia del veicolo 1.
> 1
Inserire la cilindrata del veicolo 1.
> 190.50

Inserire la tipologia del veicolo 2.
> 4
Inserire la cilindrata del veicolo 2.
> 70.50

Inserire la tipologia del veicolo 3.
> 4
Inserire la cilindrata del veicolo 3.
> 590.50

La cilindrata massima è 590.50 e la cilindrata media è 245.5.

Dopo l'elaborazione dei veicoli, ne rimangono disponibili 1.

Veicolo 0: tipologia 1, cilindrata 190.50.

A presto!
\end{verbatim}


\end{document}
