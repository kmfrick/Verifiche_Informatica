\documentclass[a4paper, 11pt]{exam}
\usepackage{amsmath}
\usepackage{amssymb}
\usepackage{amsthm}
\usepackage[italian]{babel}
\usepackage{ccicons}
\usepackage{hyperref}
\usepackage{cleveref}
\usepackage[utf8]{inputenc}
\usepackage[autostyle=false, style=english]{csquotes}
\usepackage[margin=2cm]{geometry}
\usepackage{graphicx}
\usepackage{mathrsfs}
\usepackage{multicol}
\usepackage{relsize}
\usepackage{parskip}
\usepackage{titling}
\pagestyle{plain}
\graphicspath{{./images/}}
\MakeOuterQuote{"}
\setlength{\columnseprule}{.4pt}
\renewcommand{\solutiontitle}{\noindent\textbf{R:}\enspace}
\def\dbar{{\mathchar'26\mkern-12mu d}}
\title{Verifica di Informatica - Array - Compito A}
\author{ITIS Aldini-Valeriani - Classe 3EIN}
\date{26 gennaio 2021}
\begin{document}
\maketitle
\begin{center}\fbox{\fbox{\parbox{5.5in}{

		\centering
		
		Si consegni su Classroom \textbf{un solo file} con \textbf{nome nella forma} \texttt{Nome\_Cognome\_2021-01-26\_CompitoA.cpp}.

}}}

\end{center}
%\vspace{1cm}
%Nome: \enspace \hrulefill
\vspace{1em}

\section*{Problema}

La compagnia aerea FunFlights, presa dall'entusiasmo per l'approvazione e per l'inizio della distribuzione dei vaccini per la prevenzione della COVID-19, sta facendo lavorare senza sosta i propri dipendenti per preparare la campagna pubblicitaria per l'estate 2021.

Questa campagna pubblicitaria si articolerà su tre punti: marketing aggressivo, riduzione dei prezzi e riduzione delle tratte.

La FunFlights, infatti, ha intenzione di offire solamente voli su tre tratte: Bologna-Barcellona\footnote{Capoluogo della Catalogna, non Pozzo di Gotto.}, Milano-Monaco\footnote{Di Baviera, non Montecarlo.} e Torino-Trapani.
Ognuna di queste tratte è identificata da un codice: 1 per Bologna-Barcellona, 2 per Barcellona-Bologna, 3 per Milano-Monaco, 4 per Monaco-Milano, 5 per Torino-Trapani e 6 per Trapani-Torino.

In qualità di \textit{junior software engineer} presso FunFlights ti è richiesto di realizzare un software che si interfacci con il sito italiano della compagnia.
Il programma dovrà essere realizzato nel linguaggio C++, presentare un'interfaccia da linea di comando e assolvere ai \textit{requisiti funzionali} indicati di séguito.

\begin{questions}
	\question Leggere da tastiera i prezzi e le tratte dei voli. (1 punto)
	\question Eliminare tutti i voli con destinazioni fuori dall'Italia. È fondamentale che i turisti rimangano in Italia e facciano girare l'economia del nostro Paese! (2.5 punti)
	\question Calcolare il prezzo minimo $P_{min}$ e il prezzo medio $\bar{P}$, da fornire ai \textit{data analyst} della FunFlights affinché possano effettuare analisi di mercato e preparare l'aggressiva campagna di marketing.  (2.5 punti)
	\question Calcolare il prezzo massimo $P_{max}$ ed eliminare tutti i voli con prezzo pari o superiore a $\frac{P_{max}}{2}$, in accordo con la politica di taglio dei prezzi della FunFlights. (4 punti)
\end{questions}

La manager del dipartimento di Servizi Informatici della FunFlights specifica che ogni volo è identificato da un codice numerico non negativo, ma che non è importante che il software mantenga gli stessi codici. 
Chiaramente, il software non deve cancellare voli che devono essere proposti, o inventarne di nuovi!

Un esempio di esecuzione del software è mostrato nella sezione seguente. 
Le linee che iniziano con il carattere \texttt{>} rappresentano l'input degli utenti.
Chiaramente, l'interfaccia mostrata rappresenta un mero \textit{mock-up} esemplificativo e non è assolutamente vincolante. 

Infine, la manager si raccomanda di scrivere codice pulito, funzionale e ben indentato, nel quale sia chiaro lo scopo e il funzionamento di ogni riga.
\pagebreak
\maketitle
\section*{Esempio di esecuzione}

\begin{verbatim}

Benvenuto/a in FunFlights! 

Quanti sono i voli da elaborare?
> 4

Inserire la tratta del volo 0.
> 2
Inserire il prezzo del volo 0.
> 13.99


Inserire la tratta del volo 1.
> 1
Inserire il prezzo del volo 1.
> 19.99


Inserire la tratta del volo 2.
> 4
Inserire il prezzo del volo 2.
> 11.99

Inserire la tratta del volo 3.
> 1
Inserire il prezzo del volo 3.
> 59.99

Il prezzo minimo è 11.99 e il prezzo medio è 26.49.

Dopo l'elaborazione delle tratte, rimangono disponibili 2 voli.

Volo 0: tratta 2, prezzo 13.99.
Volo 1: tratta 4, prezzo 11.99.

A presto!
\end{verbatim}


\end{document}
