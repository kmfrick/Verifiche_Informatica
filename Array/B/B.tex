\documentclass[a4paper, 11pt]{exam}
\usepackage{amsmath}
\usepackage{amssymb}
\usepackage{amsthm}
\usepackage[italian]{babel}
\usepackage{ccicons}
\usepackage{hyperref}
\usepackage{cleveref}
\usepackage[utf8]{inputenc}
\usepackage[autostyle=false, style=english]{csquotes}
\usepackage[margin=2cm]{geometry}
\usepackage{graphicx}
\usepackage{mathrsfs}
\usepackage{multicol}
\usepackage{relsize}
\usepackage{parskip}
\usepackage{titling}
\pagestyle{plain}
\graphicspath{{./images/}}
\MakeOuterQuote{"}
\setlength{\columnseprule}{.4pt}
\renewcommand{\solutiontitle}{\noindent\textbf{R:}\enspace}
\def\dbar{{\mathchar'26\mkern-12mu d}}
\title{Verifica di Informatica - Array - Compito B}
\author{ITIS Aldini-Valeriani - Classe 3EIN}
\date{26 gennaio 2021}
\begin{document}
\maketitle
\begin{center}\fbox{\fbox{\parbox{5.5in}{

		\centering

	Si consegni su Classroom \textbf{un solo file} con \textbf{nome nella forma} \texttt{Nome\_Cognome\_2021-01-26\_CompitoB.cpp}.



}}}

\end{center}
%\vspace{1cm}
%Nome: \enspace \hrulefill
\vspace{1em}

\section*{Problema}

L'azienda agricola BigFields è stata incaricata dal Governo della Repubblica Italiana di incrementare la produzione in modo da assicurare autosufficienza alimentare in caso di prolungata chiusura delle frontiere dovuta alla COVID-19.

La nuova strategia produttiva, su consiglio della \textit{chief production officer} della BigFields, verterà su due principali capisaldi: conversione all'agricoltura estensiva e \textit{data-driven analysis} della produzione.

La BigFields ha intenzione di produrre solamente quattro colture: arance, zucchine, mele e banane.
Ognuna delle colture attualmente prodotte è identificata da un codice: 1 per le arance, 2 per le zucchine, 3 per le mele, 4 per le banane, 5 per i mandarini. 
Inoltre, i campi troppo piccoli saranno da unire in campi più grossi.

In qualità di \textit{junior software engineer} presso BigFields ti è richiesto di realizzare un software che si interfacci con il terminale della \textit{chief production officer}.
Il programma dovrà essere realizzato nel linguaggio C++, presentare un'interfaccia da linea di comando e assolvere ai \textit{requisiti funzionali} indicati di séguito.

\begin{questions}
	\question Leggere da tastiera le metrature e le tipologie di colture. (1 punto)
	\question Eliminare tutte le colture di mandarini. Sono praticamente delle arance, ma più piccole! (2.5 punti)
	\question Calcolare la metratura più alta $P_{max}$ e la metratura media $\bar{P}$, da fornire ai \textit{data analyst} della BigFields affinché possano effettuare analisi di mercato e ottimizzare la produzione e la vendita.  (2.5 punti)
	\question Calcolare la metratura  minima $P_{min}$ ed eliminare tutti i campi con metratura  pari o inferiore a $2 \cdot P_{min}$, in accordo con la politica di agricoltura estensiva. (4 punti)
\end{questions}

La manager del dipartimento di Servizi Informatici della BigFields specifica che ogni campo è identificato da un codice numerico non negativo, ma che non è importante che il software mantenga gli stessi codici. 
Chiaramente, il software non deve cancellare campi che devono essere mantenuti, o inventarne di nuovi!

Un esempio di esecuzione del software è mostrato nella sezione seguente. 
Le linee che iniziano con il carattere \texttt{>} rappresentano l'input degli utenti.
Chiaramente, l'interfaccia mostrata rappresenta un mero \textit{mock-up} esemplificativo e non è assolutamente vincolante. 

Infine, la manager si raccomanda di scrivere codice pulito, funzionale e ben indentato, nel quale sia chiaro lo scopo e il funzionamento di ogni riga.
\pagebreak
\maketitle
\section*{Esempio di esecuzione}

\begin{verbatim}

Benvenuto/a in BigFields! 

Quanti sono i campi da elaborare?
> 4

Inserire la tipologia del campo 0.
> 5
Inserire la metratura (in mq) del campo 0.
> 400


Inserire la tipologia del campo 1.
> 1
Inserire la metratura (in mq) del campo 1.
> 150.75


Inserire la tipologia del campo 2.
> 2
Inserire la metratura (in mq) del campo 2.
> 100.84

Inserire la tipologia del campo 3.
> 1
Inserire la metratura (in mq) del campo 3.
> 72.45

La metratura massima è 400 e la metratura media è 181.01.

Dopo l'elaborazione dei campi, ne rimangono coltivati 1.

Campo 0: coltura 1, metratura 150.75.

A presto!
\end{verbatim}


\end{document}
