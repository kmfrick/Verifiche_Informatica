\documentclass[a4paper, 11pt]{exam}
\usepackage{amsmath}
\usepackage{amssymb}
\usepackage{amsthm}
\usepackage[italian]{babel}
\usepackage{ccicons}
\usepackage{hyperref}
\usepackage{cleveref}
\usepackage[utf8]{inputenc}
\usepackage[autostyle=false, style=english]{csquotes}
\usepackage[margin=2cm]{geometry}
\usepackage{graphicx}
\usepackage{mathrsfs}
\usepackage{multicol}
\usepackage{relsize}
\usepackage{parskip}
\usepackage{titling}
\pagestyle{plain}
\graphicspath{{./images/}}
\MakeOuterQuote{"}
\setlength{\columnseprule}{.4pt}
\renewcommand{\solutiontitle}{\noindent\textbf{R:}\enspace}
\renewcommand{\familydefault}{\sfdefault}
\def\dbar{{\mathchar'26\mkern-12mu d}}
\title{Verifica di Informatica - Strutture}
\author{ITIS Aldini-Valeriani - Classe 3EIN}
\date{25 maggio 2021}
\begin{document}
\maketitle
\begin{center}\fbox{\fbox{\parbox{5.5in}{

		\centering
		
		Si consegni su Classroom \textbf{un solo file} con \textbf{nome nella forma} \texttt{Nome\_Cognome\_2021-05-25.cpp}.

}}}

\end{center}
%\vspace{1cm}
%Nome: \enspace \hrulefill
\vspace{1em}

\section*{Problema}

La campagna vaccinale italiana contro la Covid-19, pur con i suoi intoppi, sta accelerando. 
Il Governo Italiano ha istituito una \emph{task force} di lavoratori e lavoratrici dell'informatica per ottimizzare lo smistamento e la somministrazione dei vaccini.
Nei documenti di progetto si sottolina che funzionalità prioritarie dell'infrastruttura informatica da realizzare sono l'accesso rapido ai dati dei vaccini e dei punti vaccinali.
Questo tipo di dati sono rappresentate da opportune \emph{strutture dati} definite come segue.

La struttura dati che rappresenta un \emph{lotto di vaccini} contiene i seguenti campi:
\begin{itemize}
	\item casa produttrice (e.g. AstraZeneca, Reithera, Moderna)
	\item numero di dosi di vaccino nel lotto
	\item data di produzione (nel formato GG/MM/AAAA)
\end{itemize}

La struttura dati che rappresenta un \emph{punto vaccinale} contiene i seguenti campi:
\begin{itemize}
	\item tipo di vaccino somministrato (si presuma, per semplicità, che ogni punto vaccinale somministri un solo tipo di vaccino)
	\item una stima del numero di vaccini richiesti
	\item la sigla della provincia in cui è sito (e.g. BO, RM, VV)
	\item l'indirizzo completo (e.g. "Via Sandro Bassanelli 11, Bologna")
\end{itemize}

Si implementino:
\begin{itemize}
	\item le strutture dati che rappresentano lotti di vaccini e punti vaccinali (2 punti)
	\item una funzione che legga i dati relativi a un array di lotti di vaccini, passato come argomento (2 punti)
	\item una funzione che legga i dati relativi a un array di punti vaccinali, passato come argomento (2 punti)
	\item una funzione che ordini un array di lotti vaccinali in ordine decrescente in base al numero di dosi contenute (2 punti)
		\begin{itemize}
			\item \emph{Indicazioni: } Se non si riesce a implementare questa funzione, si utilizzi quella fornita nel file allegato. 
		\end{itemize}
	\item una funzione che, ricevuto come argomento un array di lotti di vaccini, restituisca i dati di ogni lotto prodotto da una casa produttrice passata come argomento, ordinati in ordine decrescente in base al numero di dosi contenute (2 punti)
		\begin{itemize}
			\item \emph{Indicazioni:} Per filtrare i dati in base alla casa produttrice, si cicli sull'array di strutture dati in oggetto e si utilizzi una selezione (un costrutto \texttt{if}) per stampare, o comunque considerare, solamente i lotti nei quali il campo \emph{``casa produttrice''} corrisponde alla stringa passata come argomento.
Per restituire dati filtrati ordinati è sufficiente chiamare il filtro sull'array ordinato, senza dover creare un nuovo array e filtrarlo. 
		\end{itemize}
	\item una funzione che calcoli lo scarto tra vaccini richiesti prodotti da una certa casa e i vaccini effettivamente disponibili (2 punti)
		\begin{itemize}
			\item \emph{Indicazioni:} v. punto precedente
		\end{itemize}
	\item una \texttt{main()} di test che chiami queste funzioni su un campione ridotto, e.g. con 2 lotti di vaccini e 2 punti vaccinali (1 punto)
\end{itemize}

Un esempio di esecuzione del software è mostrato nella sezione seguente.
Le linee che iniziano con il carattere \texttt{>} rappresentano l'input degli utenti.
Chiaramente, l'interfaccia mostrata rappresenta un mero \textit{mock-up} esemplificativo e non è assolutamente vincolante.

La dirigente della task force si raccomanda di scrivere codice pulito, funzionale e ben indentato, nel quale sia chiaro lo scopo e il funzionamento di ogni riga.

\begin{verbatim}
Benvenut* nel software per la gestione della campagna vaccinale italiana!

Inserire il numero di lotti di vaccini.
> 3
Inserire la casa produttrice del lotto 0.
> Reithera
Inserire il numero di vaccini nel lotto 0.
> 127
Inserire la data di produzione del lotto 0.
> 01/05/2021
Inserire la casa produttrice del lotto 1.
> AstraZeneca
Inserire il numero di vaccini nel lotto 1.
> 254
Inserire la data di produzione del lotto 1.
> 15/06/2021
Inserire la casa produttrice del lotto 2.
> Reithera
Inserire il numero di vaccini nel lotto 2.
> 100
Inserire la data di produzione del lotto 2.
> 21/04/2021

Inserire il numero di punti vaccinali. 
> 2
Inserire la casa produttrice dei vaccini somministrati dal punto 0. 
> Reithera
Inserire una stima del numero di vaccini richiesti dal punto 0.
> 200
Inserire la sigla della provincia nella quale è sito il punto 0.
> BO
Inserire l'indirizzo del punto 0.
> Via Lenin 23, Bologna
Inserire la casa produttrice dei vaccini somministrati dal punto 0. 
> Pfizer
Inserire una stima del numero di vaccini richiesti dal punto 0.
> 100
Inserire la sigla della provincia nella quale è sito il punto 0.
> MI
Inserire l'indirizzo del punto 0.
> Via Stalingrado 32, Milano

L'array di vaccini ordinato è:
0: {AstraZeneca, 254, 15/06/2021}
1: {Reithera, 127, 01/05/2021}
2: {Reithera, 100, 21/04/2021}

Inserire la casa produttrice per la chiamata di test al filtro:
> Reithera
Risultato:
0: {Reithera, 127, 01/05/2021}
1: {Reithera, 100, 21/04/2021}

Inserire la casa produttrice per il calcolo dello scarto:
> Reithera
Risultato:
Sono richiesti 200 vaccini Reithera. 227 vaccini sono disponibili.
Scarto: 27

A presto! Andrà tutto bene :)
\end{verbatim}
\end{document}
