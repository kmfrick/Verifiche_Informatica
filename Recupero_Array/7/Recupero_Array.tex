\documentclass[a4paper, 11pt]{exam}
\usepackage{amsmath}
\usepackage{amssymb}
\usepackage{amsthm}
\usepackage[italian]{babel}
\usepackage{ccicons}
\usepackage{hyperref}
\usepackage{cleveref}
\usepackage[utf8]{inputenc}
\usepackage[autostyle=false, style=english]{csquotes}
\usepackage[margin=2cm]{geometry}
\usepackage{graphicx}
\usepackage{mathrsfs}
\usepackage{multicol}
\usepackage{relsize}
\usepackage{parskip}
\usepackage{titling}
\pagestyle{plain}
\graphicspath{{./images/}}
\MakeOuterQuote{"}
\setlength{\columnseprule}{.4pt}
\renewcommand{\solutiontitle}{\noindent\textbf{R:}\enspace}
\def\dbar{{\mathchar'26\mkern-12mu d}}
\renewcommand{\familydefault}{\sfdefault}
\title{Verifica di Informatica - Recupero Array - Compito B}
\author{ITIS Aldini-Valeriani - Classe 3EIN}
\date{25 febbraio 2021}
\begin{document}
\maketitle
\begin{center}\fbox{\fbox{\parbox{5.5in}{

		\centering
		
		Si consegni su Classroom \textbf{un solo file} con \textbf{nome nella forma} \texttt{Cognome\_Nome\_2021-02-25\_CompitoB.cpp}.

}}}

\end{center}
%\vspace{1cm}
%Nome: \enspace \hrulefill
\vspace{1em}

\section*{Problema}

Si realizzi un programma scritto nel linguaggio C++ che presenti un'interfaccia da linea di comando e assolva ai \textit{requisiti funzionali} indicati di séguito.

\begin{questions}
	\question Leggere da tastiera un array di numeri con la virgola $V$ (2 punti)
	\question Calcolare l'elemento massimo $V_{max}$ e la sua posizione e stampare entrambi i dati a schermo. (2 punti)
	\question Calcolare la media $V_{avg}$ dell'array $V$ e stamparla a schermo. (2 punti)
	\question Calcolare il numero di elementi maggiori della media $V_{avg}$. (1 punto)
\end{questions}

Un esempio di esecuzione del software è mostrato nella sezione seguente. 
Le linee che iniziano con il carattere \texttt{>} rappresentano l'input degli utenti.
Chiaramente, l'interfaccia mostrata rappresenta un mero \textit{mock-up} esemplificativo e non è assolutamente vincolante. 

Si raccomanda di scrivere codice pulito, funzionale e ben indentato, nel quale sia chiaro lo scopo e il funzionamento di ogni riga.
\section*{Esempio di esecuzione}

\begin{verbatim}
Inserire la dimensione dell'array V.
> 4

Inserire l'elemento 0.
> 11.1

Inserire l'elemento 1.
> 32.2

Inserire l'elemento 2.
> 48.2

Inserire l'elemento 3.
> 0.4

L'elemento massimo è 48.2, in posizione 2. La media è 22.975.
Ci sono 2 elementi superiori alla media.

A presto!
\end{verbatim}


\end{document}
