\documentclass[a4paper, 11pt]{exam}
\usepackage{amsmath}
\usepackage{amssymb}
\usepackage{amsthm}
\usepackage[italian]{babel}
\usepackage{ccicons}
\usepackage{hyperref}
\usepackage{cleveref}
\usepackage[utf8]{inputenc}
\usepackage[autostyle=false, style=english]{csquotes}
\usepackage[margin=2cm]{geometry}
\usepackage{graphicx}
\usepackage{mathrsfs}
\usepackage{multicol}
\usepackage{relsize}
\usepackage{parskip}
\usepackage{titling}
\pagestyle{plain}
\graphicspath{{./images/}}
\MakeOuterQuote{"}
\setlength{\columnseprule}{.4pt}
\renewcommand{\solutiontitle}{\noindent\textbf{R:}\enspace}
\def\dbar{{\mathchar'26\mkern-12mu d}}
\renewcommand{\familydefault}{\sfdefault}
\title{Verifica di Informatica - Recupero Array - Compito A}
\author{ITIS Aldini-Valeriani - Classe 3EIN}
\date{25 febbraio 2021}
\begin{document}
\maketitle
\begin{center}\fbox{\fbox{\parbox{5.5in}{

		\centering
		
		Si consegni su Classroom \textbf{un solo file} con \textbf{nome nella forma} \texttt{Cognome\_Nome\_2021-02-25\_CompitoA.cpp}.

}}}

\end{center}
%\vspace{1cm}
%Nome: \enspace \hrulefill
\vspace{1em}

\section*{Problema}

Il birrificio Morroni sta intensificando la produzione di birra artigianale in modo da poter più efficientemente servire un mercato in rapida espansione.

Per ottimizzare il parco macchinari dell'azienda è necessario un accurato inventario delle tipologie di birra prodotte in grado di interfacciarsi con l'ERP utilizzato dal reparto produzione.
In qualità di \emph{junior software engineer} presso Morroni Srl ti è richiesto di realizzare tale software.

La manager del dipartimento di Servizi Informatici del birrificio specifica che ogni birra prodotta è identificata da un codice numerico non negativo, da una \emph{tipologia} e da una \emph{gradazione alcolica}.
Il codice numerico identifica la posizione all'interno del database dell'azienda, implementato come un array statico. 
La tipologia è un numero intero, mentre la gradazione alcolica un numero con la virgola.
Le birre \emph{lager} sono identificate dal codice 100, le \emph{stout} dal codice 101, le \emph{India Pale Ale (IPA)} dal codice 102 e le \emph{bock} dal codice 103.

Il programma dovrà essere realizzato nel linguaggio C++, presentare un'interfaccia da linea di comando e assolvere ai \textit{requisiti funzionali} indicati di séguito.

\begin{questions}
	\question Leggere da tastiera le tipologie e le gradazioni delle birre. (3 punti)
	\question Calcolare la gradazione alcolica massima $V_{max}$ e stamparla a schermo, identificando anche la tipologia di birra corrispondente. (2 punti)
	\question Calcolare la gradazione alcolica media $V_{bock}$ tra le birre \emph{bock}. (3 punti)
	\question Calcolare il \emph{prezzo} di ogni birra secondo la seguente regola: per tutte le birre tranne le \emph{stout} il prezzo a bottiglia è $\frac{1}{5}$ della gradazione alcolica; per le \emph{stout} il prezzo a bottiglia è $\frac{1}{3}$ della gradazione alcolica. (2 punti)
\end{questions}

Un esempio di esecuzione del software è mostrato nella sezione seguente. 
Le linee che iniziano con il carattere \texttt{>} rappresentano l'input degli utenti.
Chiaramente, l'interfaccia mostrata rappresenta un mero \textit{mock-up} esemplificativo e non è assolutamente vincolante. 

Infine, la manager si raccomanda di scrivere codice pulito, funzionale e ben indentato, nel quale sia chiaro lo scopo e il funzionamento di ogni riga.
\pagebreak
\maketitle
\section*{Esempio di esecuzione}

\begin{verbatim}

Benvenuto/a nel birrificio Morroni! 

Quanti sono le birre da elaborare?
> 4

Inserire la tipologia della birra 0.
> 101
Inserire la gradazione della birra 0.
> 11.2

Inserire la tipologia della birra 1.
> 100
Inserire la gradazione della birra 1.
> 4.7

Inserire la tipologia della birra 2.
> 103
Inserire la gradazione della birra 2.
> 8.1

Inserire la tipologia della birra 3.
> 103
Inserire la gradazione della birra 3.
> 7.9

La gradazione alcolica massima è 11.2, per una birra di tipologia 101.

La gradazione alcolica media tra le birre bock è 8.0.

Il prezzo della birra 0 è 3.73 a bottiglia.
Il prezzo della birra 1 è 0.94 a bottiglia.
Il prezzo della birra 2 è 1.62 a bottiglia.
Il prezzo della birra 3 è 1.58 a bottiglia.

A presto!
\end{verbatim}


\end{document}
